% This is samplepaper.tex, a sample chapter demonstrating the
% LLNCS macro package for Springer Computer Science proceedings;
% Version 2.20 of 2017/10/04
%
\documentclass[runningheads]{llncs}
%
\newif\ifFinal
\Finalfalse  % set to true for camera ready

\usepackage{graphicx}
% Used for displaying a sample figure. If possible, figure files should
% be included in EPS format.
%
% If you use the hyperref package, please uncomment the following line
% to display URLs in blue roman font according to Springer's eBook style:
% \renewcommand\UrlFont{\color{blue}\rmfamily}

\begin{document}
%
\title{One Implementation to Rule them all}
%
%\titlerunning{Abbreviated paper title}
% If the paper title is too long for the running head, you can set
% an abbreviated paper title here
%
%\author{First Author\inst{1}\orcidID{0000-1111-2222-3333} \and
%Second Author\inst{2,3}\orcidID{1111-2222-3333-4444}}
%
%\authorrunning{F. Author et al.}
% First names are abbreviated in the running head.
% If there are more than two authors, 'et al.' is used.
%
%\institute{Princeton University, Princeton NJ 08544, USA \and
%Springer Heidelberg, Tiergartenstr. 17, 69121 Heidelberg, Germany
%\email{lncs@springer.com}\\
%\url{http://www.springer.com/gp/computer-science/lncs} \and
%ABC Institute, Rupert-Karls-University Heidelberg, Heidelberg, Germany\\
%\email{\{abc,lncs\}@uni-heidelberg.de}}
%

\ifFinal
\author{First Author\inst{1}\orcidID{0000-1111-2222-3333} \and
Second Author\inst{2,3}\orcidID{1111-2222-3333-4444} \and
Third Author\inst{3}\orcidID{2222--3333-4444-5555}}
%
\authorrunning{F. Author et al.}
% First names are abbreviated in the running head.
% If there are more than two authors, 'et al.' is used.
%
\institute{Princeton University, Princeton NJ 08544, USA \and
Springer Heidelberg, Tiergartenstr. 17, 69121 Heidelberg, Germany
\email{lncs@springer.com}\\
\url{http://www.springer.com/gp/computer-science/lncs} \and
ABC Institute, Rupert-Karls-University Heidelberg, Heidelberg, Germany\\
\email{\{abc,lncs\}@uni-heidelberg.de}}
%
\else
\author{Anonymised}

\maketitle              % typeset the header of the contribution
%
\begin{abstract}
  Semiring, a fundamental concept in abstract algebra, plays
  a pivotal role in graph theory, optimisation, social choice, voting, 
  and many more. Its generality puts it into a sweet spot 
  where a single abstract implementation can be used, depending 
  on the concrete instantiation of abstract operators, 
  for shortest path, the Schulze method, program analysis, 
  databases, networking, etc. 

  In this paper, we present a formal implementation of semiring in the Coq 
  theorem prover and demonstrate its usability by instantiating it
  with various operators to 
  get shortest path implementation, Schulze method, more examples. 

  


  
\keywords{First keyword  \and Second keyword \and Another keyword.}
\end{abstract}

\section{Introduction}

Semiring is a generalisation of ring where the additive inverse requirement is dropped. 
It is equipped with 

is an algebraic structures that 


combines the essential properties of both rings and monoids. 
They are equipped with two binary operations, typically addition and multiplication, that exhibit 
intriguing properties, making them a powerful tool in solving a diverse range of problems. 
This abstract explores the algebraic properties of semirings, highlighting their distributive, associative, and zero elements.

\section{Schulze Method}

\section{Technical Details (Explain the whole semiring)}

\section{Case Studies}
  \begin{itemize}
    \item Shortest Path
    \item Optimisation 
    \item Put everything that can be computed using our algorithm
  \end{itemize}

\section{Related Work and Limitations}
%
%
%
%
% ---- Bibliography ----
%
% BibTeX users should specify bibliography style 'splncs04'.
% References will then be sorted and formatted in the correct style.
%
\bibliographystyle{splncs04}
\bibliography{ref}
%

\end{document}
